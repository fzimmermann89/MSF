% das Papierformat zuerst
%\documentclass[a4paper, 11pt]{article}

% deutsche Silbentrennung
%\usepackage[ngerman]{babel}

% wegen deutschen Umlauten
%\usepackage[ansinew]{inputenc}

% hier beginnt das Dokument
%\begin{document}


\thispagestyle{empty}

\begin{figure}[t]

 \includegraphics[width=0.3\textwidth]{abb/misc/TULogo.eps}
~~~~~~~~~~
\end{figure}


\begin{verbatim}


\end{verbatim}

\begin{center}
\Large{Technische Universit\"at Berlin}\\
\end{center}


\begin{center}
\Large{Fakult\"at II Mathematik und Naturwissenschaften}
\end{center}
\begin{verbatim}


\end{verbatim}
\begin{center}
\doublespacing
\textbf{\LARGE{\titleDocument}}\\
\singlespacing
\begin{verbatim}

\end{verbatim}
\textbf{{~\subjectDocument}}
\end{center}
\begin{verbatim}

\end{verbatim}
\begin{center}

\end{center}
\begin{verbatim}





\end{verbatim}
\begin{flushleft}
\begin{tabular}{llll}
\textbf{Autoren:} &  Halgurd Taher &\\& Felix Zimmermann& \\
&  Paul-Rainer Affeld & \\
& & \\
\textbf{Version vom:}  & \today &
\end{tabular}
\end{flushleft}