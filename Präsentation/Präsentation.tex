\documentclass[xcolor=dvipsnames]{beamer} 
\useoutertheme{infolines} 
\usetheme{CambridgeUS} 
\setbeamertemplate{items}[ball] 
\setbeamertemplate{blocks}[rounded][shadow=true] 
\setbeamertemplate{navigation symbols}{} 

\usepackage[ngerman]{babel}
\usepackage[latin1]{inputenc}
\usepackage{wrapfig}
\usepackage{caption}
\usepackage{subcaption}
\addtobeamertemplate{block begin}{\setlength{\textwidth}{0.9\textwidth}}{}

\usepackage[style=numeric, sorting=none]{biblatex} 
\usepackage[justification=raggedright,singlelinecheck=false,labelfont=bf]{caption}
\addbibresource{citations.bib}%the bibliography database 
\DeclareFieldFormat{journaltitle}{#1}         
%\DeclareFieldFormat{month}{}           
\DeclareFieldFormat[article]{title}{}
\DeclareFieldFormat[article]{volume}{\mkbibbold{#1}\addcomma\space} 
\DeclareFieldFormat[article]{number}{\space #1}
\DeclareFieldFormat{url}{} 
\DeclareFieldFormat{doi}{}
\DeclareFieldFormat{pages}{}
\renewcommand*{\multinamedelim}{\addcomma\space}
\renewcommand*{\finalnamedelim}{\addcomma\space} 
\renewcommand*{\labelnamepunct}{\addcomma\space}
\renewcommand*{\bibpagespunct}{\adddot\space}
\renewbibmacro{in:}{}

\title[Synchronisation in Netzwerken\hspace{12mm} \insertframenumber/
\inserttotalframenumber]{Synchronisation in Netzwerken:\\Master Stability Function und
Permutationssymmetrien}
\institute[]{Institut fur Theoretische Physik, Technische Universitat Berlin, Germany}



\author[F. Zimmermann, H. Taher, P. Affeld]{
	\underline{Felix Zimmermann, Halgurd Taher, Paul-Rainer Affeld}
}

\date[\today]{\today}

\renewcommand{\familydefault}{times}
\renewcommand{\rmdefault}{times}



\begin{document}

\frame{
	\titlepage
	
	
	\begin{figure}
		\includegraphics[width=0.12\textwidth]{tulogo.pdf}
	\end{figure}
}


\frame{
	\frametitle{Inhalt}
	\tableofcontents
}

\section{Einleitung}
\frame{
	\frametitle{Table of Contents}
	\tableofcontents[currentsection]

}

\frame{
	\frametitle{Dynamik auf Netzwerken}
		\begin{itemize}
		\item $N$ miteinander gekoppelte Knoten
		\item Jeder Knoten wird durch dynamische Gleichung beschrieben
		\end{itemize}
	\begin{align}\label{eq:dyneqcommon}
					\overset{\cdot}{\boldsymbol{x}}_i(t)&=\boldsymbol{f}(\boldsymbol{x}_i(t))+\sigma\sum_j A_{ij}\boldsymbol{h}(\boldsymbol{x}_j)
					\\\notag & i=1,...,N
					\\\notag & A_{ij}\text{ Kopplungsmatrix}
					\\\notag & \boldsymbol{f},\boldsymbol{h}:\mathbb{R}^n\rightarrow\mathbb{R}^n
	\end{align}
	Definiere
	\begin{align}
	\boldsymbol{X}=\left(\boldsymbol{x}_1,...,\boldsymbol{x}_N\right)^{\text{T}},
	\boldsymbol{F}=\left(\boldsymbol{f}(\boldsymbol{x}_1),...,\boldsymbol{f}(\boldsymbol{x}_N)\right)^{\text{T}},
	\boldsymbol{H}=\left(\boldsymbol{h}(\boldsymbol{x}_1),...,\boldsymbol{h}(\boldsymbol{x}_N)\right)^{\text{T}}
	\end{align}
	$\Rightarrow$
	\begin{align}
	\overset{\cdot}{\boldsymbol{X}}=
	\end{align}


%	\begin{equation}\label{eq:dyneqcommonall}
%				\overset{\cdot}{\boldsymbol{ x}}(t)=
%				\left[\sum_m^M\boldsymbol{E}^{(m)}\otimes D\boldsymbol{F}(\boldsymbol{s}_m(t))+
%				\sigma\boldsymbol{A}\sum_m^M\boldsymbol{E}^{(m)}\otimes D\boldsymbol{H}(\boldsymbol{s}_m(t))\right]
%				\boldsymbol{\delta x}(t)
%	\end{equation}
%	\begin{equation}\label{eq:delx}
%				\overset{\cdot}{\boldsymbol{\delta x}}(t)=
%				\left[\sum_m^M\boldsymbol{E}^{(m)}\otimes D\boldsymbol{F}(\boldsymbol{s}_m(t))+
%				\sigma\boldsymbol{A}\sum_m^M\boldsymbol{E}^{(m)}\otimes D\boldsymbol{H}(\boldsymbol{s}_m(t))\right]
%				\boldsymbol{\delta x}(t)
%	\end{equation}
	

	

}

\frame{
	\frametitle{Beispiel}
	Diskretes System mit $N=11$ Knoten
	(TODO Gleichung f�r xt+1)
	(beta, sigma Parameter)

}

\frame{
	\frametitle{Synchronit�t}
	Synchronit�t : $x1(t)=x2(t)...s(t)$
	\begin{itemize}
	\item tmp
	\end{itemize}
	Stabilit�t der Synchronit�t :\\
	Wie entwickelt sich kleine Abweichung von s(t) zeitlich weiter?\\
	Master Stability Equation:
	Formel deltaX (ohne Cluster).... mit Einheitsmatrix Kronecker...
	Master Stability Function (gr��ter Ljapunow Exponent gro� lambda)  
    gro� lambda=limes t$->$inf summe $ln abs(deltax_i(t+1))/deltax_i(t))$
    lambda>0 instabil, fehler w�chst, bahnkurven xi entfernen sich von s(t)
    lambda<0 stabil, fehler schrumpft, bahnkurven xi n�hern sich wieder s(t)
}
\frame{
	\frametitle{Synchronit�t}
   	Voraussetzung :\\
   	Bei Synchronit�t jeder Knoten gleicher Input $->$ u.a Konstante Zeilensumme von A \\
}

\frame{
	\frametitle{Synchronit�t}
   	Voraussetzung :\\
   	Bei Synchronit�t jeder Knoten gleicher Input $->$ u.a Konstante Zeilensumme von A \\
   	(AIJ aus mittlerer simulation)
   	Zeilensumme NICHT konstant
   
}

\frame{
	\frametitle{Synchronit�t}
	Simulation des Beispiels mit 
   	(AIJ aus mittlerer simulation)
   
}
\frame{
	\frametitle{Synchronit�t}
	- Keine globale synchronisation
	- 5 gruppen von knoten die sich synchron verhalten
	$->$ Cluster
   (evtl screenshot)
}

\frame{
	\frametitle{Cluster}
	Knoten eines Clusters k�nnen bei Synchronit�t gleichen Input haben.
   (da Gleiche Zeilensumme innerhalb eines Clusters)
   Knoten k�nnen vertauscht werden, dynamik bleibt gleich
   Netzwerk besitzt offensichtlich symmetrien
   Suche nach permutationssymmetrien 
   mathematisch beschrieben durch permutationsmatrizen $P_i$
   mit $A=PAP^-1 -> A$ bleibt unver�ndert bei tauschen der knoten
   
   
}
\frame{
	\frametitle{Cluster}
    zwei Arten von Permutationen:
    1. tausche Knoten A und B $->$ A und B gleiches Cluster
    2. tausche Knoten A und B sowie C und D $->$ Cluster (AB) und (CD) "verschr�nkt"
}

\frame{
	\frametitle{Cluster}
	Clustersuche in der Regel numerisch
	Bibliothek nauty  $[1]$
}

\frame{
	\frametitle{Formen von Synchronit�t}
-globale Synchronit�t
-isolierte Synchronit�t innerhalb eines Clusters
-gemeinsame Synchronit�t zweier verschr�nkter Cluster
	(AB)synchron und (CD)synchron
}
\frame{
	\frametitle{Isolierte Synchronit�t}
Warum kann ein Cluster Syncrhon sein w�hrend andere nicht synchron sind?
(pecora argumentation mit permutationsmatrizen vorbeiziehen und gleichen input f�r das cluster)
}
\frame{
	\frametitle{Stabilt�t der Clustersynchronit�t}
Stabilit�tsanalyse
	f�r einzelne Cluster schwierig: in welche richtung sollte deltax betrachtet werden?
	->Basistransformation mit Transformationsmatrix T
	T blockdiagonalisiert A, sodass linker oberer MxM  Block die Bewegung innerhalb der Synchronisationsmannigfaltigkeit beschreibt
	 an gleichung xxx von links  TxIn dranmultiplizieren
	(gleichungen f�r eta) Str�ung in neuer basis)
	
	aus dieser MSE kann gr��ter Ljapunow-Exponent f�r  jedes Cluster berechnet werden
}
\frame{
	\frametitle{Simulation}
Stabilit�tsanalyse
}

\frame{
	\frametitle{Fazit}
In Netzwerken mit  Symmetrien existieren Cluster
Cluster k�nnen Synchron laufen
Stabilti�tsanalyse durch Basistransformation m�glich}
\section{asd}
\frame{
	\frametitle{Inhalt}
	\tableofcontents[currentsection]
}


\frame{
	\vfill
	\begin{center}
		Thank you for your attention.
	\end{center}
	\vfill
}

\end{document}