\section{Synchronisation}
\subsection*{Globale Synchronisation}
Das Netzwerk wird als global synchron bezeichnet, wenn sich alle Variablen $\boldsymbol{x}_i$ zeitlich gleich verhalten.
\begin{align*}
\boldsymbol{x}_1(t)=\boldsymbol{x}_2(t)=...=\boldsymbol{x}_N(t)=:\boldsymbol{s}(t)
\end{align*}
Es spielt dabei keine Rolle, ob dieses Verhalten z.B konstant, periodisch oder chaotisch ist.
Voraussetzung für die Existenz globaler Synchronisation ist eine konstante Zeilensumme der Kopplungsmatrix $\boldsymbol{A}$. Liegt keine konstante Zeilensumme vor, so erhalten zwei Knoten von einer unterschiedlicher Anzahl Knoten Input. Seien nun alle Knoten zu einem Zeitpunkt $t$ synchron, so entwickeln sich durch diesen unterschiedlichen Input die beiden Knoten zeitlich unterschiedlich und die Synchronisation würde zwangsläufig aufgehoben.
\subsection*{Isolierte Synchronisation}
Isolierte Synchronisation liegt vor, wenn eine Gruppe von Knoten (Cluster) oben genanntes Verhalten aufweist, während ein anderer Teil des Netzwerks nicht synchron mit dieser Gruppe ist. Voraussetzung ist (analog zur globalen Synchronisation) eine konstante Zeilensumme innerhalb dieser Gruppe.

