\section{Synchronisation}
\subsection*{Globale Synchronisation}
Das Netzwerk wird als global synchron bezeichnet, wenn sich alle Variablen $\boldsymbol{x}_i$ zeitlich gleich verhalten.
\begin{align*}
\boldsymbol{x}_1(t)=\boldsymbol{x}_2(t)=...=\boldsymbol{x}_N(t)=:\boldsymbol{s}(t)
\end{align*}
Es spielt dabei keine Rolle, ob dieses Verhalten z.B konstant, periodisch oder chaotisch ist.

\subsection*{Isolierte Synchronisation}
Isolierte Synchronisation liegt vor, wenn eine Gruppe von Knoten oben genanntes Verhalten aufweist, während ein anderer Teil des Netzwerks nicht synchron mit dieser Gruppe ist.

