\section{Einleitung}\label{einleitung}
Dynamische Netzwerke spielen in der heutigen Wissenschaft eine wichtige Rolle. So lassen sich beispielsweise Prozesse im Gehirn zwischen Neuronen über Netzwerke beschreiben und analysieren. Großflächige Stromnetze stellen ebenfalls ein klassisches Beispiel eines Netzwerkes dar. Es ist von großem Interesse, Prozesse in solchen Systemen hinsichtlich ihrer Dynamik und Stabilität zu untersuchen. Ein bekanntes Hilfmittel zur Analyse von Netzwerken ist die sogenannte Master Stability Function (MSF), mit deren Hilfe sich Aussagen über die Stabilität von globalen Synchronisationszuständen treffen lassen.\\
Bei der Betrachtung von real existierenden Netzwerken können allerdings (häufiger als globale Synchronisation) Cluster aus synchronen Knoten (Nervenzellen, Kraftwerke) beobachtet werden. So spielt bei verscheidenen Erkrankungen, wie fokaler Epilepsie, das Auftreten synchroner Areale im Gehirn eine entscheidende Rolle zur Pathogenese.\\
Ziel dieser Ausarbeitung ist es, eine Methode zu präsentieren und zu verifizieren mit der nicht nur eine globale Analyse des Netzwerkes möglich ist, sondern auch die Clusterbildung und lokales Verhalten dieser Cluster untersucht werden kann.