\section{Fazit}\label{fazit}
Die Existenz von synchronisierenden Clustern in einem Netzwerk ergibt sich aus vorhandenen Symmetrien des Netzwerkes. Das Phänomen der isolierten Synchronisation einzelner Cluster lässt sich über die Betrachtung der Dynamik unter Symmetriepermutationen erklären. 
Die Master Stability Function ist ein bekanntes Hilfsmittel um die globale Synchronisation in Netzwerken hinsichtlich der Stabilität zu analysieren. Durch eine Basistransformation können mit der MSF darüber hinaus Aussagen über die Stabilität der lokalen Synchronisation innerhalb der Cluster getroffen werden.\\
Anhand eines Simulationsbeispiels wurden verschiedene Netzwerke mit diskreter Dynamik auf Permutationssymmetrien untersucht, darüber die vorhandenen Cluster identifiziert und die vorgestellte Methode zur Analyse der Stabilität der Synchronisation verifiziert.
