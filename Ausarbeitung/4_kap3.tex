\section{Stabilität der Synchronisation}
Ein synchroner Zustand eines Netzwerks lässt sich hinsichtlicht seiner Stabilität untersuchen. Dabei wird eine kleine Abweichung der Anfangsbedingungen des synchronen Zustandes angenommen und berechnet, wie sich diese Störung zeitlich weiterentwickelt.
\begin{align}\label{eq:deltaxi}
\delta \boldsymbol{x}_i(t) &= \boldsymbol{x}_i - \boldsymbol{s}(t)\\
\delta \overset{\cdot}{\boldsymbol{x}}_i(t) &= \boldsymbol{f}(\boldsymbol{x}_i(t))+\sigma\sum_j A_{ij}\boldsymbol{h}\left(\boldsymbol{x}_j(t)\right) - \overset{\cdot}{\boldsymbol{s}}(t)
\end{align}
Eine Linearisierung dieser Gleichung um die Bahnkurve $\boldsymbol{s}(t)$ liefert in der Kronecker-Produkt Schreibweise die Master Stability Equation (MSE) in Gleichung (\ref*{eq:mse}).
\begin{align}\label{eq:mse}
\delta\overset{\cdot}{\boldsymbol{X}}(t)=
\left[D\boldsymbol{F}(\boldsymbol{s}(t))+\sigma\boldsymbol{A}\otimes D\boldsymbol{H}(\boldsymbol{s}(t))\right]\delta\boldsymbol{X}(t)
\end{align}

\subsection*{Ljapunow-Exponenten}
Die Ljapunow-Exponenten beschreiben, wie weit sich Bahnkurven für große Zeiten von einander entfernen, verglichen mit der Abweichung zum Zeitpunkt $t=0$. Eine mögliche Definition ist in Gleichung (\ref{eq:ljapunow}) gegeben.
\begin{align}\label{eq:ljapunow}
\lambda_i=\lim_{t\rightarrow\infty}\frac{1}{t} ln\left(\frac{|\delta \boldsymbol{x}_i(t)|}{|\delta \boldsymbol{x}_i(0)|}\right)
\end{align}
Wird die Abweichung zu $\delta\boldsymbol{x}_i(0)$ größer, so ist der Quotient größer als 1, sonst kleiner. Es gilt also folgende Unterscheidung für die Werte der Ljapunow-Exponenten.
\begin{equation}
\begin{cases}
\lambda_i < 0, \text{ Abweichung vom synchronen Zustand verschwindet}\\
\lambda_i > 0, \text{ Abweichung vom synchronen Zustand wächst}
\end{cases}
\end{equation}
Positive Ljapunow-Exponenten weisen eine instabile Synchronisation nach und negative eine stabile. Der Stabilitätsbegriff bezieht sich hierbei auf Invarianz der Bahnkurven gegenüber Änderungen der Anfangsbedingungen (für lange Zeiten). 
\subsection*{Master Stability Function}
Da es für Instabilität genügt, wenn einer der Exponenten größer als 0 ist, ist es sinnvoll nur den größten zu betrachten. Dieser wird Master Stability Function (MSF) gennant. Die MSF hat dabei als Parameter die Eigenschaften des Systems, z.b. die Kopplungsstärke $\sigma$.



