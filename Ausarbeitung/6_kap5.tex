\section{Simulationsbeispiel}
Die vorgestellten Methoden zur lokalen Stabilitätsanalyse eines Netzwerks werden in diesem Abschnitt auf ein Beispiel angewendet. Bisher wurden nur dynamische Systeme mit kontinuierlichen Variablen betrachtet. Die zugrunde liegende Theorie lässt sich auch auf diskrete Systeme anwenden. Dabei ist die Dynamik nicht durch ein Differentialgleichungssystem gegeben, sondern durch eine Iterationsvorschrift. Das hier betrachtete Netzwerk \cite{pecora2014} folgt der Dynamik in Gleichung (\ref{eq:bspdyn}).
\begin{align}
	x_i^{t+1}&=\left[\beta\mathcal{I}(x_i^t)+\sigma \sum_j^N A_{ij}\mathcal{I}(x_j^t)\right] \text{mod}\quad 2\pi
	\\\notag & \beta,\sigma \text{ Kopplungsparameter}
	\\\notag &\mathcal{I}(x)=\frac{1-Cos(x)}{2}
\end{align}\label{eq:bspdyn}
Das Netzwerk ist symmetrisch und besteht aus $N=11$ Knoten.