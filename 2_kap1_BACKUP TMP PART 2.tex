\subsection{Zweiteilchen Schrödingergleichung im Halbleiter}\label{kapitel12}
\#Herleitung fehlt\\ 
\#Auf Confinement, Unordnungspotential, Nanostruktur Dimension muss noch eingegangen werden\\
Die Zweiteilchen-Wellenfunktion $\psi$ eines Elektron im Leitungsband mit der Koordinate $\boldsymbol{r}_e$ und ein Loch im Valenzband mit der Koordinate $\boldsymbol{r}_h$ im Coulombpotential erfüllt die stationäre Gleichung (\ref{eq:statex}),
\begin{equation}\label{eq:statex}
\left( 
-\frac{\hbar^2\Delta_{\boldsymbol{r}_e}}{2m_e} 
-\frac{\hbar^2\Delta_{\boldsymbol{r}_h}}{2m_h} 
-\frac{1}{4\pi \epsilon_0\epsilon_r |\boldsymbol{r_e}-\boldsymbol{r_e}|}
\right)\psi(\boldsymbol{r_e},\boldsymbol{r_h})=
\left(
E-E_G-E_{exc}
\right)\psi(\boldsymbol{r_e},\boldsymbol{r_h})
\end{equation}
wobei $E_G$ die Bandlückenenergie ist und $E_{exc}$ die Austauschenergie.

Es ist zweckmäßig dies in der Dirac-Notation durchzuführen.
Daher wird folgende Definition eingeführt:
\begin{equation}
|\boldsymbol{k}_e,\boldsymbol{k}_h>(\boldsymbol{r}_e,\boldsymbol{r}_h)=
\end{equation}